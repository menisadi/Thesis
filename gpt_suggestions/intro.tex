Machine learning, in general, is a field combining statistics and algorithm design. 
The main goal is to try and leverage information in order to produce systems and software capable of performing diverse tasks. 
This field has experienced tremendous growth in recent years. 
Systems that grow out of machine learning find many and quite varied uses: 
from analyzing medical tests, to chatbots and automatic art generators, to autonomous vehicles. 
Most of the involvement in the field, both applied and research, revolves around the construction of sophisticated models and new tools that will allow the industry to continue fast-forward toward more goals and peaks. 
At the same time, there is great interest in understanding the limitations of the core concepts and possibilities inherent in each tool and in each situation. To this end, we must create a precise mathematical system that will define the situations and challenges we face and enable their systematic and meticulous research.

The field of learning theory is a fundamental pillar of machine learning, providing a rigorous framework for understanding the mechanisms of learning algorithms and their ability to acquire knowledge from data. 
To fully explore the concepts and challenges of learning theory, it is essential to establish a solid theoretical foundation. Key concepts include hypothesis spaces, empirical risk minimization, and the bias-variance trade-off. Theoretical properties such as convergence guarantees, sample complexity, and computational complexity provide insights into the performance and limitations of learning algorithms. 

In the context of learning theory, the notion of compression has gained considerable attention. Compression refers to the ability to succinctly represent and describe patterns or regularities in data. A system that exhibits compression can convey its results more concisely than merely detailing the results themselves. The connection between compression and learning has been established through theories and results developed in statistical learning.

By considering these various aspects, including the foundations, challenges, and connections to compression, we can gain a deeper understanding of the limitations and possibilities of learning algorithms. This understanding forms the basis for our investigation into the specific questions and goals outlined in this thesis.
