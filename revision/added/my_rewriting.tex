In the last decades we have experienced continuous growth in all areas of research in computer science. Many tools and algorithms have been developed and are still being developed to analyze inputs of different types and to understand what is possible and what is impossible in each and every field.
At the same time, the world of static research is also experiencing a significant boom and concepts such as "statistical significance", "correlation" and "linear regression" have become a must in the toolbox of natural and social science researchers.
Another new field in which interest has grown over the years is "signal processing", the research field that integrates the two worlds of static and dynamic research. In this field, the goal is to analyze a signal, which is a function of time, and to understand its properties and characteristics. The signal can be of any type, such as a sound signal, a video signal, a signal that represents the temperature in a given area, and more. 
With the development of wide range of communication technologies the need for signal processing has become more and more important.
On this background the field of "machine learning" has developed. 
Machine learning is a field of computer science that deals with the development of algorithms that can learn from data and make predictions on new data.
