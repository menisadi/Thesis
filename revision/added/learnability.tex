Addressing the question of learnability necessitates the development of rigorous mathematical frameworks. One such framework is based on the notion of uniform convergence. 
The principle of uniform convergence asserts that as the size of the training dataset grows, the algorithm's performance on unseen data should converge to its expected performance.
In other words, the algorithm should generalize well to new instances beyond the training set. The study of uniform convergence provides insights into the trade-offs between the complexity of a learning algorithm, the size of the training dataset, and the algorithm's ability to generalize accurately.

Another approach to the learnability question is to formulate learning problems as optimization problems. 
Convex optimization provides a powerful mathematical framework for solving optimization problems where the objective function and constraints exhibit convexity.
By casting learning problems as optimization problems, researchers can leverage this rich mathematical theory and develop learning algorithms, allowing for efficient computation and convergence guarantees to globally optimal solutions.
The interplay between convex optimization and machine learning has led through the years to significant advancements in developing robust and scalable learning algorithms.

However, even with tools such as those of uniform convergence and convex optimization at our disposal, determining the learnability of a given problem remains a formidable challenge. 
It is often influenced by the inherent complexity of the problem domain, the quality and quantity of available data, and the expressiveness of the learning algorithm. 
This challenge has prompted researchers to explore additional avenues for understanding the limits and capabilities of machine learning systems.

One such avenue is rooted in the principle of Occam's Razor
